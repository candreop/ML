%
% Glossary entries with associated acronym entry
%

% AI
\newglossaryentry{aig}{
  name={artificial intelligence},
  description={
    The capacity of computers or other machines to exhibit or simulate intelligent behaviour
  }
}
\newglossaryentry{ai}{
  type=\acronymtype,
  name={AI},
  description={Artificial Intelligence},
  first={Artificial Intelligence (AI)\glsadd{aig}},
  see=[Glossary:]{aig}
}

% CNN
\newglossaryentry{cnng}{
  name={convolutional neural network},
  plural={convolutional neural networks},
  description={
  }
}
\newglossaryentry{cnn}{
  type=\acronymtype,
  name={CNN},
  description={Convolutional Neural Network},
  first={Convolutional Neural Network (CNN)\glsadd{cnng}},
  plural={CNNs},
  see=[Glossary:]{cnng}
}

% ML
\newacronym {ml} {ML} {Machine Learning}

% DL
\newacronym {dl} {DL} {Deep Learning}

% SVM
\newglossaryentry{svmg}{
  name={support vector machine},
  plural={support vector machines},
  description={
  }
}
\newglossaryentry{svm}{
  type=\acronymtype,
  name={SVM},
  description={Support Vector Machine},
  first={Support Vector Machine (SVM)\glsadd{svmg}},
  plural={SVMs},
  see=[Glossary:]{svmg}
}

% ILSVRC
\newglossaryentry{ilsvrcg}{
  name={support vector machine},
  plural={support vector machines},
  description={
    An annual competition in the field of computer vision.
    See also \gls{ImageNet}
  }
}
\newglossaryentry{ilsvrc}{
  type=\acronymtype,
  name={ILSVRC},
  description={ImageNet Large Scale Visual Recognition Challenge},
  first={ImageNet Large Scale Visual Recognition Challenge (ILSVRC)\glsadd{ilsvrcg}},
  see=[Glossary:]{ilsvrcg}
}

%
% Glossary entries without associated acronym entry
%

\newglossaryentry{0/1 loss function}{
  name={0/1 loss function},
  description={
  }
}

\newglossaryentry{AlexNet}{
  name={AlexNet},
  description={
  }
}

\newglossaryentry{activation function}{
  name=activation function,
  description={
  }
}

\newglossaryentry{cell-plane}{
  name={cell-plane},
  plural={cell-planes},
  description={
    A group of cells within a layer of 
    \gls{neuron}-like S-cells or C-cells in a \gls{neocognitron}.
  }
}

\newglossaryentry{classification}{
  name={classification},
  description={
  }
}

\newglossaryentry{convolution}{
  name={convolution},
  description={
  }
}

\newglossaryentry{complex cell}{
  name={complex cell},
  plural={complex cells},
  description={
  }
}

\newglossaryentry{cross-correlation}{
  name={cross-correlation},
  description={
  }
}

\newglossaryentry{cerebral cortex}{
  name={cerebral cortex},
  description={
  }
}

\newglossaryentry{epoch}{
  name={epoch},
  description={
  }
}

\newglossaryentry{feature engineering}{
  name=feature engineering,
  description={
    Using domain knowledge to extract features of raw data
  }
}

\newglossaryentry{Inception}{
  name={Inception},
  description={
  }
}

\newglossaryentry{GoogLeNet}{
  name={GoogLeNet},
  description={
    Also known as \gls{Inception}
  }
}

\newglossaryentry{gradient descent}{
  name={gradient descent},
  hyphenated={gradient-descent},
  description={
  }
}

\newglossaryentry{hinge loss}{
  name={hinge loss},
  description={
  }
}

\newglossaryentry{ImageNet}{
  name={ImageNet},
  description={
    A database with more than 14 million hand-annotated images
    containing more than 20,000 distinct object categories.
  }
}

\newglossaryentry{least squares}{
  name={least squares},
  description={
  }
}

\newglossaryentry{linear model}{
  name={linear model},
  plural={linear models},
  description={
    Define a linear model
  }
}

\newglossaryentry{loss function}{
  name={loss function},
  description={
  }
}

\newglossaryentry{LeNet-5}{
  name={LeNet-5},
  description={
    One of the first convolutional network architectures.
  }
}

\newglossaryentry{McCulloch-Pitts neuron}{
  name=McCulloch-Pitts neuron,
  description={
    A single-layer perceptron
  }
}

\newglossaryentry{Mark 1 perceptron}{
  name=Mark 1 perceptron,
  description={
    The original, custom-built hardware implementation of perceptron
  }
}

\newglossaryentry{neocognitron}{
  name=neocognitron,
  description={
  }
}

\newglossaryentry{neuron}{
  name=neuron,
  description={
  }
}

\newglossaryentry{objective function}{
  name={objective function},
  description={
  }
}

\newglossaryentry{perceptron}{
  name=perceptron,
  description={
    A perceptron is the simplest neural network. 
    It contains a single input layer and an output node
  }
}

\newglossaryentry{perceptron criterion}{
  name={perceptron criterion},
  description={
  }
}

\newglossaryentry{primary visual cortex}{
  name={primary visual cortex},
  description={
   The area of the \gls{visual cortex} that receives sensory input from
   the \gls{lateral geniculate nucleus}. 
   See also 
   \gls{visual area 1 (V1)}, 
   \gls{Brodmann area 17}, and 
   \gls{striate cortex}
  }
}
% ---- alternative terms of the above
\newglossaryentry{visual area 1 (V1)}{
  name={visual area 1 (V1)},
  description={
   Another term for the \gls{primary visual cortex} (see that entry for explanation)
  }
}
\newglossaryentry{Brodmann area 17}{
  name={Brodmann area 17},
  description={
   Another term for the \gls{primary visual cortex} (see that entry for explanation) 
  }
}
\newglossaryentry{striate cortex}{
  name={striate cortex},
  description={
   Another term for the \gls{primary visual cortex} (see that entry for explanation) 
  }
}
% ----

\newglossaryentry{regression}{
  name={regression},
  description={
    A technique that relates a dependent variable to one or more independent variables.
  }
}

\newglossaryentry{ResNet}{
  name={ResNet},
  description={
  }
}

\newglossaryentry{simple cell}{
  name={simple cell},
  plural={simple cells},
  description={
  }
}

\newglossaryentry{stochastic}{
  name={stochastic},
  description={
  }
}

\newglossaryentry{supervised learning}{
  name={supervised learning},
  description={
  }
}

\newglossaryentry{visual cortex}{
  name={visual cortex},
  description={
   The area of the \gls{cerebral cortex} that processes visual information.
  }
}

\newglossaryentry{VGG}{
  name={VGG},
  description={
  }
}

\newglossaryentry{lateral geniculate nucleus}{
  name={lateral geniculate nucleus},
  description={
   A structure in the thalamus of the mammalian brain and a key element 
   of the visual pathway. It connects the thalamus with the optic nerve.
  }
}

\newglossaryentry{ZFNet}{
  name={ZFNet},
  description={
  }
}

%
% Notable people
%

\newglossaryentry{Fukushima}{
  type=people,
  name={K.Fukushima},
  sort={Fukushima, Kunihiko},
  description={
    Japanese computer scientist who developed the \gls{neocognitron} 
    inspired by the biological \gls{primary virtual cortex} studies of
    \gls{Hubel} and \gls{Wiesel}.
  }
}

\newglossaryentry{Hubel}{
  type=people,
  name={D.Hubel},
  sort={Hubel, David},
  description={
  }
}

\newglossaryentry{Rosenblatt}{
  type=people,
  name={F.Rosenblatt},
  sort={Rosenblatt, Frank},
  description={
    American psychologist known in the field of Artificial Intelligence. 
    Sometimes called the father of Deep Learning.
    Built a hardware implementation of the perceptron
  }
}

\newglossaryentry{Wiesel}{
  type=people,
  name={T.Wiesel},
  sort={Wiesel, Torsten},
  description={
  }
}


