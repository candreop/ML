
%
% Glossary entries with associated acronym entry
%

% AD
\newglossaryentry{adg}{
  name={automatic differentiation},
  description={
    A technique to evaluate the derivative of a function specified by a \gls{computation graph},
    which is easily programmable, efficient and numerically stable.
    See also \gls{algorithmic differentiation} and \gls{computational differentiation}.
  }
}
\newglossaryentry{ad}{
  type=\acronymtype,
  name={AD},
  description={Automatic Differentiation},
  first={Automatic Differentiation (AD)\glsadd{adg}},
  see=[Glossary:]{adg}
}

% AI
\newglossaryentry{aig}{
  name={artificial intelligence},
  description={
    The capacity of computers or other machines to exhibit or simulate intelligent behaviour
  }
}
\newglossaryentry{ai}{
  type=\acronymtype,
  name={AI},
  description={Artificial Intelligence},
  first={Artificial Intelligence (AI)\glsadd{aig}},
  see=[Glossary:]{aig}
}

% ANI
\newglossaryentry{anig}{
  name={artificial narrow intelligence},
  description={
   A type of \gls{ai} with a narrow range of capabilities.
   Also referred to as \gls{weak ai} or \gls{narrow ai}
  }
}
\newglossaryentry{ani}{
  type=\acronymtype,
  name={ANI},
  description={Artificial Narrow Intelligence},
  first={Artificial Narrow Intelligence (ANI)\glsadd{anig}}
}

% AGI
\newglossaryentry{agig}{
  name={artificial general intelligence},
  description={
   A type of \gls{ai} with human-level capabilities.
   Also referred to as \glsadd{strong ai} or \glsadd{deep ai}
  }
}
\newglossaryentry{agi}{
  type=\acronymtype,
  name={AGI},
  description={Artificial General Intelligence},
  first={Artificial General Intelligence (AGI)\glsadd{agig}}
}

% ASI
\newglossaryentry{asig}{
  name={artificial superintelligence},
  description={
   A type of \gls{ai} that greatly exceeds human capabilities
  }
}
\newglossaryentry{asi}{
  type=\acronymtype,
  name={ASI},
  description={Artificial Superintelligence},
  first={Artificial Superintelligence (ASI)\glsadd{asig}}
}

% CIFAR
\newglossaryentry{cifarg}{
  name={CIFAR},
  description={
    Canadian Institute For Advanced Research. 
    Its Neural Computation & Adaptive Perception program brought together 
    computer scientists, neuroscientists, biologists, physicists, and others,
    and is credited with major advances in \gls{ML}.
    Many datasets commonly used to train \gls{ML} algorithms are named after CIFAR.
    See \gls{CIFAR-10}, \gls{CIFAR-100}
  }
}
\newglossaryentry{cifar}{
  type=\acronymtype,
  name={CIFAR},
  description={Canadian Institute For Advanced Research},
  first={Canadian Institute For Advanced Research (CIFAR)\glsadd{cifarg}},
  see=[Glossary:]{cifarg}
}

% CNN
\newglossaryentry{cnng}{
  name={convolutional neural network},
  plural={convolutional neural networks},
  description={
  }
}
\newglossaryentry{cnn}{
  type=\acronymtype,
  name={CNN},
  description={Convolutional Neural Network},
  first={Convolutional Neural Network (CNN)\glsadd{cnng}},
  plural={CNNs},
  see=[Glossary:]{cnng}
}

% CPU
\newglossaryentry{cpug}{
  name={CPU},
  description={
    A specialised electronic circuit 
    which controls the execution of instructions
    in a computer system.
  }
}
\newglossaryentry{cpu}{
  type=\acronymtype,
  name={CPU},
  description={Central Processing Unit},
  first={Central Processing Unit (CPU)\glsadd{cpug}},
  plural={CPUs},
  firstplural={Central Processing Units (CPUs)},
  see=[Glossary:]{cpug}
}


% DL
\newglossaryentry{dlg}{
  name={deep learning},
  description={
  }
}
\newglossaryentry{dl}{
  type=\acronymtype,
  name={DL},
  description={Deep Learning},
  first={Deep Learning (DL)\glsadd{dlg}},
  see=[Glossary:]{dlg}
}

% DL
\newglossaryentry{gang}{
  name={generative adversarial network},
  description={
  }
}
\newglossaryentry{gan}{
  type=\acronymtype,
  name={GAN},
  description={Generative Adversarial Network},
  first={Generative Adversarial Network (GAN)\glsadd{gang}},
  plural={GANs},
  firstplural={Generative Adversarial Networks (GANs)},
  see=[Glossary:]{gang}
}

% % VGG
% \newglossaryentry{vggg}{
%   name={Visual Geometry Group},
%   description={
%     A research group in Oxford.
%   }
% }
% \newglossaryentry{vgg}{
%   type=\acronymtype,
%   name={VGG},
%   description={Visual Geometry Group},
%   first={Visual Geometry Group (VGG)\glsadd{vggg}},
%   see=[Glossary:]{vggg}
% }

% GPU
\newglossaryentry{gpug}{
  name={GPU},
  description={
    A specialised electronic circuit with a parallel structure that
    makes them more efficient than \glspl{CPU} for processing large blocks of data in parallel.
  }
}
\newglossaryentry{gpu}{
  type=\acronymtype,
  name={GPU},
  description={Graphics Processing Unit},
  first={Graphics Processing Unit (GPU)\glsadd{gpug}},
  plural={GPUs},
  firstplural={Graphics Processing Units (GPUs)},
  see=[Glossary:]{gpug}
}

% ILSVRC
\newglossaryentry{ilsvrcg}{
  name={ILSVRC},
  description={
    An annual competition in the field of computer vision.
    See also \gls{ImageNet}
  }
}
\newglossaryentry{ilsvrc}{
  type=\acronymtype,
  name={ILSVRC},
  description={ImageNet Large Scale Visual Recognition Challenge},
  first={ImageNet Large Scale Visual Recognition Challenge (ILSVRC)\glsadd{ilsvrcg}},
  see=[Glossary:]{ilsvrcg}
}

% ML
\newglossaryentry{mlg}{
  name={machine learning},
  description={
  }
}
\newglossaryentry{ml}{
  type=\acronymtype,
  name={ML},
  description={Machine Learning},
  first={Machine Learning (ML)\glsadd{mlg}},
  see=[Glossary:]{mlg}
}

% % MNIST
% \newglossaryentry{mnistg}{
%   name={MNIST},
%   description={
%   }
% }
% \newglossaryentry{mnist}{
%   type=\acronymtype,
%   name={MNIST},
%   description={Modified National Institute of Standards and Technology},
%   first={Modified National Institute of Standards and Technology (MNIST)\glsadd{mnistg}},
%   see=[Glossary:]{mnistg}
% }


% ReLU
\newglossaryentry{relug}{
  name={rectified linear unit},
  plural={rectified linear units},
  description={
    A type of simple and piecewise linear \gls{activation function} 
    (0 for negative values of its argument, x, and equal to x otherwise) that,
    nevertheless, introduces the property of non-linearity to a \gls{dlg} model.
  }
}
\newglossaryentry{relu}{
  type=\acronymtype,
  name={ReLU},
  description={Rectified Linear Unit},
  first={Rectified Linear Unit (ReLU)\glsadd{relug}},
  plural={ReLUs},
  see=[Glossary:]{relug}
}

% SNARC
\newglossaryentry{snarcg}{
  name={SNARC},
  description={
    Stochastic Neural Analog Reinforcement Calculator. 
    The first neural net machine built by \gls{Minsky} and \gls{Edmonds}
  }
}
\newglossaryentry{snarc}{
  type=\acronymtype,
  name={SNARC},
  description={Stochastic Neural Analog Reinforcement Calculator},
  first={Stochastic Neural Analog Reinforcement Calculator (SNARC)\glsadd{snarcg}},
  see=[Glossary:]{snarcg}
}

% SVM
\newglossaryentry{svmg}{
  name={support vector machine},
  plural={support vector machines},
  description={
  }
}
\newglossaryentry{svm}{
  type=\acronymtype,
  name={SVM},
  description={Support Vector Machine},
  first={Support Vector Machine (SVM)\glsadd{svmg}},
  plural={SVMs},
  see=[Glossary:]{svmg}
}

% ----------------------------------------------------------------------------------

%
% Acronyms without associated glossary entry
%


% MSE
\newglossaryentry{mse}{
  type=\acronymtype,
  name={MSE},
  description={Mean Squared Error},
  first={Mean Squared Error (MSE)}
}

% ----------------------------------------------------------------------------------

%
% Glossary entries without associated acronym entry
%

\newglossaryentry{0/1 loss function}{
  name={0/1 loss function},
  description={
  }
}

\newglossaryentry{activation function}{
  name=activation function,
  description={
  }
}

\newglossaryentry{activation map}{
  name={activation map},
  description={
    Also known as \gls{feature map}
  }
}

\newglossaryentry{AlexNet}{
  name={AlexNet},
  description={
    A \glspl{cnn} architecture.
    Winner of \gls{ilsvrc} for 2012
  }
}

\newglossaryentry{algorithmic differentiation}{
  name={algorithmic differentiation},
  description={
    Another term for \gls{adg}
  }
}

\newglossaryentry{array programming}{
  name={array programming},
  description={
    An alternative term for \gls{vectorization}
  }
}

\newglossaryentry{average pooling}{
  name={average pooling},
  description={
  }
}

\newglossaryentry{back propagation}{
  name={back propagation},
  hyphenated={back-propagation},
  description={
    A special case of the Reverse Mode of \gls{ad} 
    used by the \gls{ml} community
  }
}

\newglossaryentry{cell-plane}{
  name={cell-plane},
  plural={cell-planes},
  description={
    A group of cells within a layer of 
    \gls{neuron}-like S-cells or C-cells in a \gls{neocognitron}
  }
}

\newglossaryentry{CIFAR-10}{
  name={CIFAR-10},
  description={
    A labelled collection of 60k 32x32 colour images
    assembled by \gls{Krizhevsky}, \gls{Nair}, and \gls{Hinton}
    It includes 6k images for each of 10 distinct classes
    (airplanes, cars, birds, cats, deer, dogs, frogs, horses, ships, and trucks)
  }
}

\newglossaryentry{CIFAR-100}{
  name={CIFAR-100},
  description={
    A labelled collection of 60k 32x32 colour images similar to \gls{CIFAR-10}.
    It includes 600 images for each of 100 distinct classes.
  }
}

\newglossaryentry{classification}{
  name={classification},
  description={
  }
}

\newglossaryentry{computational differentiation}{
  name={computational differentiation},
  description={
    Another term for \gls{adg}.
  }
}

\newglossaryentry{computation graph}{
  name={computation graph},
  description={
  }
}

\newglossaryentry{convolution}{
  name={convolution},
  description={
  }
}

\newglossaryentry{convolutional layer}{
  name={convolutional layer},
  plural={convolutional layers},
  description={
  }
}

\newglossaryentry{complex cell}{
  name={complex cell},
  plural={complex cells},
  description={
  }
}

\newglossaryentry{cross-correlation}{
  name={cross-correlation},
  description={
  }
}

\newglossaryentry{cerebral cortex}{
  name={cerebral cortex},
  description={
  }
}

\newglossaryentry{deep ai}{
  name={deep AI},
  description={
  Another term for artificial general intelligence (\gls{agi})
  }
}

\newglossaryentry{epoch}{
  name={epoch},
  description={
  }
}

\newglossaryentry{feature engineering}{
  name=feature engineering,
  description={
    Using domain knowledge to extract features of raw data
  }
}

\newglossaryentry{feature map}{
  name={feature map},
  description={
    Also known as \gls{activation map}
  }
}

\newglossaryentry{feed forward}{
  name={feed forward},
  hyphenated={feed-forward},
  description={
  }
}

\newglossaryentry{filter}{
  name={filter},
  plural={filters},
  description={
    See also \gls{kernel}
  }
}

\newglossaryentry{full padding}{
  name={full padding},
  description={
  }
}

\newglossaryentry{Jacobian matrix}{
  name={Jacobian matrix},
  description={
    The Jacobian matrix $J$ of a function $f: \mathbb{R}^n \rightarrow \mathbb{R}^m$,
    is a $m \times n$ matrix of all its first order derivatives
  }
}

\newglossaryentry{Inception}{
  name={Inception},
  description={
    A \glspl{cnn} architecture.
    Its v1 was the winner of \gls{ilsvrc} for 2014.
    Also known as \gls{GoogLeNet}
  }
}

\newglossaryentry{GoogLeNet}{
  name={GoogLeNet},
  description={
    A \glspl{cnn} architecture.
    Winner of \gls{ilsvrc} for 2014.
    Also known as \gls{Inception} (v1)
  }
}

\newglossaryentry{gradient descent}{
  name={gradient descent},
  hyphenated={gradient-descent},
  description={
  }
}

\newglossaryentry{half padding}{
  name={half padding},
  description={
  }
}

\newglossaryentry{hinge loss}{
  name={hinge loss},
  description={
  }
}

\newglossaryentry{ImageNet}{
  name={ImageNet},
  description={
    A database with more than 14 million hand-annotated images
    containing more than 20,000 distinct object categories
  }
}

\newglossaryentry{kernel}{
  name={kernel},
  plural={kernels},
  description={
    See also \gls{filter}.
  }
}

\newglossaryentry{lateral geniculate nucleus}{
  name={lateral geniculate nucleus},
  description={
   A structure in the thalamus of the mammalian brain and a key element 
   of the visual pathway. It connects the thalamus with the optic nerve
  }
}

\newglossaryentry{least squares}{
  name={least squares},
  description={
  }
}

\newglossaryentry{limited memory ai}{
  name={limited memory AI},
  description={
    Task-specific \gls{ai} system that 
    can store memories for a limited time and learn from experience
  }
}

\newglossaryentry{linear model}{
  name={linear model},
  plural={linear models},
  description={
  }
}

\newglossaryentry{loss function}{
  name={loss function},
  description={
  }
}

\newglossaryentry{LeNet-5}{
  name={LeNet-5},
  description={
    The first convolutional network architecture.
  }
}

\newglossaryentry{max pooling}{
  name={max pooling},
  description={
  }
}

\newglossaryentry{mean squared error loss function}{
  name={\gls{mse} loss function},
  description={
  }
}

\newglossaryentry{McCulloch-Pitts neuron}{
  name=McCulloch-Pitts neuron,
  description={
    A single-layer perceptron
  }
}

\newglossaryentry{Mark 1 perceptron}{
  name=Mark 1 perceptron,
  description={
    The original, custom-built hardware implementation of perceptron
  }
}

\newglossaryentry{narrow ai}{
  name={narrow AI},
  description={
  Another term for artificial narrow intelligence (\gls{ani})
  }
}

\newglossaryentry{neocognitron}{
  name=neocognitron,
  description={
  }
}

\newglossaryentry{neuron}{
  name=neuron,
  description={
  }
}

\newglossaryentry{objective function}{
  name={objective function},
  description={
  }
}

\newglossaryentry{padding}{
  name={padding},
  description={
  }
}

\newglossaryentry{perceptron}{
  name=perceptron,
  description={
    A perceptron is the simplest neural network. 
    It contains a single input layer and an output node
  }
}

\newglossaryentry{perceptron criterion}{
  name={perceptron criterion},
  description={
  }
}

\newglossaryentry{pooling}{
  name={pooling},
  description={
  }
}

\newglossaryentry{primary visual cortex}{
  name={primary visual cortex},
  description={
   The area of the \gls{visual cortex} that receives sensory input from
   the \gls{lateral geniculate nucleus}. 
   See also 
   \gls{visual area 1 (V1)}, 
   \gls{Brodmann area 17}, and 
   \gls{striate cortex}
  }
}
% ---- alternative terms of the above
\newglossaryentry{visual area 1 (V1)}{
  name={visual area 1 (V1)},
  description={
   Another term for the \gls{primary visual cortex} (see that entry for explanation)
  }
}
\newglossaryentry{Brodmann area 17}{
  name={Brodmann area 17},
  description={
   Another term for the \gls{primary visual cortex} (see that entry for explanation) 
  }
}
\newglossaryentry{striate cortex}{
  name={striate cortex},
  description={
   Another term for the \gls{primary visual cortex} (see that entry for explanation) 
  }
}
% ----

\newglossaryentry{reactive ai}{
  name={reactive AI},
  description={
    Task-specific \gls{ai} system that reacts to inputs in predictable manner,
    but does not store memories and does not learn from experience
  }
}

\newglossaryentry{regression}{
  name={regression},
  description={
    A technique that relates a dependent variable to one or more independent variables
  }
}

\newglossaryentry{ResNet}{
  name={ResNet},
  description={
    A \glspl{cnn} architecture.
    Winner of \gls{ilsvrc} for 2015.
  }
}

\newglossaryentry{self-awareness ai}{
  name={self-awareness AI},
  description={
    Hypothetical \gls{ai} system with human-level artificial consciousness
  }
}

\newglossaryentry{simple cell}{
  name={simple cell},
  plural={simple cells},
  description={
  }
}

\newglossaryentry{stochastic}{
  name={stochastic},
  description={
  }
}

\newglossaryentry{stochastic pooling}{
  name={stochastic pooling},
  description={
  }
}

\newglossaryentry{stride}{
  name={stride},
  description={
  }
}

\newglossaryentry{strong ai}{
  name={strong AI},
  description={
  Another term for artificial general intelligence (\gls{agi})
  }
}

\newglossaryentry{sub-sampling}{
  name={subsampling},
  description={
  }
}

\newglossaryentry{sum pooling}{
  name={sum pooling},
  description={
  }
}

\newglossaryentry{supervised learning}{
  name={supervised learning},
  description={
  }
}

\newglossaryentry{theory of mind ai}{
  name={theory of mind AI},
  description={
    Emotionally intelligent, next-level \gls{ai} system able to learn any task
  }
}

\newglossaryentry{valid padding}{
  name={valid padding},
  description={
  }
}

\newglossaryentry{vectorization}{
  name={vectorization},
  description={
  }
}

\newglossaryentry{visual cortex}{
  name={visual cortex},
  description={
   The area of the \gls{cerebral cortex} that processes visual information.
  }
}

\newglossaryentry{VGG}{
  name={VGG},
  description={
    A \glspl{cnn} architecture. A close second of \gls{ilsvrc} for 2013.
  }
}

\newglossaryentry{weak ai}{
  name={weak AI},
  description={
  Another term for artificial narrow intelligence (\gls{ani})
  }
}

\newglossaryentry{ZFNet}{
  name={ZFNet},
  description={
    A \glspl{cnn} architecture.
    Winner of \gls{ilsvrc} for 2013.
  }
}

% ----------------------------------------------------------------------------------

%
% Notable people
%

\newglossaryentry{Bengio}{
  type=people,
  name={Y.Bengio},
  sort={Bengio, Yoshua},
  description={
  }
}

\newglossaryentry{Edmonds}{
  type=people,
  name={D.Edmonds},
  sort={Edmonds, Dean},
  description={
  }
}

\newglossaryentry{Fukushima}{
  type=people,
  name={K.Fukushima},
  sort={Fukushima, Kunihiko},
  description={
    Japanese computer scientist who developed the \gls{neocognitron} 
    inspired by the biological \gls{primary visual cortex} studies of
    \gls{Hubel} and \gls{Wiesel}
  }
}

\newglossaryentry{Hinton}{
  type=people,
  name={G.Hinton},
  sort={Hinton, Geoffrey},
  description={
  }
}

\newglossaryentry{Hubel}{
  type=people,
  name={D.Hubel},
  sort={Hubel, David},
  description={
  }
}

\newglossaryentry{Krizhevsky}{
  type=people,
  name={A.Krizhevsky},
  sort={Krizhevsky, Alex},
  description={
  }
}

\newglossaryentry{LeCun}{
  type=people,
  name={Y.LeCun},
  sort={LeCun, Yann},
  description={
  }
}

\newglossaryentry{McCulloch}{
  type=people,
  name={W.McCulloch},
  sort={McCulloch, Warren},
  description={
  }
}

\newglossaryentry{Minsky}{
  type=people,
  name={M.Minsky},
  sort={Minsky, Marvin},
  description={
  }
}

\newglossaryentry{Nair}{
  type=people,
  name={V.Nair},
  sort={Nair, Vinod},
  description={
  }
}

\newglossaryentry{Pitts}{
  type=people,
  name={W.Pitts},
  sort={Pitts, Walter},
  description={
  }
}

\newglossaryentry{Rosenblatt}{
  type=people,
  name={F.Rosenblatt},
  sort={Rosenblatt, Frank},
  description={
    American psychologist known in the field of \gls{ai}. 
    Sometimes called the father of \gls{dl}.
    Built a hardware implementation of the perceptron
  }
}

\newglossaryentry{Rumelhart}{
  type=people,
  name={D.E.Rumelhart},
  sort={Rumelhart, David Everett},
  description={
    American psychologist known for several contributions in the
    field of mathematical psychology and \gls{ai}.
    Together with \gls{Hinton} and \gls{Williams},
    he published a highly influential paper on \gls{back propagation}
  }
}

\newglossaryentry{Turing}{
  type=people,
  name={A.M.Turing},
  sort={Turing, Alan},
  description={
  }
}

\newglossaryentry{Wiesel}{
  type=people,
  name={T.Wiesel},
  sort={Wiesel, Torsten},
  description={
  }
}

\newglossaryentry{Williams}{
  type=people,
  name={R.J.Williams},
  sort={Williams, Ronald J.},
  description={
  }
}

