%
% Glossary / Entries starting from A
%
% Prof. Costas Andreopoulos <c.andreopoulos@cern.ch>
% ~~~~~~~~~~~~~~~~~~~~~~~~~~~~~~~~~~~~~~~~~~~~~~~~~~~~~~~~~~~~~~~~~~~~~~~~~~~~~
%

% AD
\newglossaryentry{adg}{
  name={automatic differentiation},
  description={
    A technique to evaluate the derivative of a function specified by a \gls{computation graph},
    which is easily programmable, efficient and numerically stable.
    See also \gls{algorithmic differentiation} and \gls{computational differentiation}.
  }
}
\newglossaryentry{ad}{
  type=\acronymtype,
  name={AD},
  description={Automatic Differentiation},
  first={Automatic Differentiation (AD)\glsadd{adg}},
  see=[Glossary:]{adg}
}

% AI
\newglossaryentry{aig}{
  name={artificial intelligence},
  description={
    The capacity of computers or other machines to exhibit or simulate intelligent behaviour
  }
}
\newglossaryentry{ai}{
  type=\acronymtype,
  name={AI},
  description={Artificial Intelligence},
  first={Artificial Intelligence (AI)\glsadd{aig}},
  see=[Glossary:]{aig}
}

% ANI
\newglossaryentry{anig}{
  name={artificial narrow intelligence},
  description={
   A type of \gls{ai} with a narrow range of capabilities.
   Also referred to as \gls{weak ai} or \gls{narrow ai}
  }
}
\newglossaryentry{ani}{
  type=\acronymtype,
  name={ANI},
  description={Artificial Narrow Intelligence},
  first={Artificial Narrow Intelligence (ANI)\glsadd{anig}}
}

% AGI
\newglossaryentry{agig}{
  name={artificial general intelligence},
  description={
   A type of \gls{ai} with human-level capabilities.
   Also referred to as \glsadd{strong ai} or \glsadd{deep ai}
  }
}
\newglossaryentry{agi}{
  type=\acronymtype,
  name={AGI},
  description={Artificial General Intelligence},
  first={Artificial General Intelligence (AGI)\glsadd{agig}}
}

% ASI
\newglossaryentry{asig}{
  name={artificial superintelligence},
  description={
   A type of \gls{ai} that greatly exceeds human capabilities
  }
}
\newglossaryentry{asi}{
  type=\acronymtype,
  name={ASI},
  description={Artificial Superintelligence},
  first={Artificial Superintelligence (ASI)\glsadd{asig}}
}

% activation function
\newglossaryentry{activation function}{
  name={activation function},
  plural={activation functions},
  description={
  }
}

% activation map
\newglossaryentry{activation map}{
  name={activation map},
  description={
    Also known as \gls{feature map}
  }
}

% adaptive subgradient
\newglossaryentry{adaptive subgradient}{
  name={adaptive subgradient},
  description={
    A \gls{gradient descent}-based algorithm using 
    adaptive \glspl{learning rate}
  }
}

% AdaGrad
\newglossaryentry{AdaGradg}{
  name={Adaptive Gradient},
  description={
    An \gls{adaptive subgradient} algorithm
  }
}
\newglossaryentry{AdaGrad}{
  type=\acronymtype,
  name={AdaGrad},
  description={Adaptive Gradient Algorithm},
  first={Adaptive Gradient Algorithm (AdaGrad)\glsadd{AdaGradg}},
  see=[Glossary:]{AdaGradg}
}

% Adam
\newglossaryentry{Adamg}{
  name={Adaptive Moments},
  description={
    An \gls{adaptive subgradient} algorithm
  }
}
\newglossaryentry{Adam}{
  type=\acronymtype,
  name={Adam},
  description={Adaptive Moments Algorithm},
  first={Adaptive Moments Algorithm (Adam)\glsadd{Adamg}},
  see=[Glossary:]{Adamg}
}

% AlexNet
\newglossaryentry{AlexNet}{
  name={AlexNet},
  description={
    A \glspl{cnn} architecture.
    Winner of \gls{ilsvrc} for 2012
  }
}

% algorithmic differentiation
\newglossaryentry{algorithmic differentiation}{
  name={algorithmic differentiation},
  description={
    Another term for \gls{adg}
  }
}

% anomaly detection
\newglossaryentry{anomaly detection}{
  name={anomaly detection},
  description={
    In this \gls{ml} task, a model processes a set of data examples
    and flags some of them as being unusual.
  }
}

% array programming
\newglossaryentry{array programming}{
  name={array programming},
  description={
    An alternative term for \gls{vectorization}
  }
}

% average pooling
\newglossaryentry{average pooling}{
  name={average pooling},
  description={
  }
}
