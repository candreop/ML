%
% Glossary / Entries starting from C
%
% Prof. Costas Andreopoulos <c.andreopoulos@cern.ch>
% ~~~~~~~~~~~~~~~~~~~~~~~~~~~~~~~~~~~~~~~~~~~~~~~~~~~~~~~~~~~~~~~~~~~~~~~~~~~~~
%

% capacity
\newglossaryentry{capacity}{
  name={capacity},
  description={
    The ability of a \gls{ml} model to describe a broad variety of 
    data-generating processes
  }
}

% cell-plane
\newglossaryentry{cell-plane}{
  name={cell-plane},
  plural={cell-planes},
  description={
    A group of cells within a layer of 
    \gls{neuron}-like S-cells or C-cells in a \gls{neocognitron}
  }
}

% cerebral cortex
\newglossaryentry{cerebral cortex}{
  name={cerebral cortex},
  description={
  }
}

% CIFAR
\newglossaryentry{cifarg}{
  name={CIFAR},
  description={
    Canadian Institute For Advanced Research. 
    Its Neural Computation & Adaptive Perception program brought together 
    computer scientists, neuroscientists, biologists, physicists, and others,
    and is credited with major advances in \gls{ML}.
    Many datasets commonly used to train \gls{ML} algorithms are named after CIFAR.
    See \gls{CIFAR-10}, \gls{CIFAR-100}
  }
}
\newglossaryentry{cifar}{
  type=\acronymtype,
  name={CIFAR},
  description={Canadian Institute For Advanced Research},
  first={Canadian Institute For Advanced Research (CIFAR)\glsadd{cifarg}},
  see=[Glossary:]{cifarg}
}

% CIFAR-10
\newglossaryentry{CIFAR-10}{
  name={CIFAR-10},
  description={
    A labelled collection of 60k 32x32 colour images
    assembled by \gls{Krizhevsky}, \gls{Nair}, and \gls{Hinton}
    It includes 6k images for each of 10 distinct classes
    (airplanes, cars, birds, cats, deer, dogs, frogs, horses, ships, and trucks)
  }
}

% CIFAR-100
\newglossaryentry{CIFAR-100}{
  name={CIFAR-100},
  description={
    A labelled collection of 60k 32x32 colour images similar to \gls{CIFAR-10}.
    It includes 600 images for each of 100 distinct classes.
  }
}

% classification
\newglossaryentry{classification}{
  name={classification},
  description={
    In this \gls{ml} task, a \gls{ml} model 
    is asked to find in which of $k$ categories 
    some $n$-dimensional input belongs to.
    The model {\bf discovers a function 
    $f: \mathbb{R}^n \rightarrow \{1,\dots,k\}$}
    that assigns the category $y = f(\vect{x})$ 
    of the input $\vect{x} \in \mathbb{R}^n$
  }
}

% classifier
\newglossaryentry{classifier}{
  name={classifier},
  description={
  }
}

% CNN
\newglossaryentry{cnng}{
  name={convolutional neural network},
  plural={convolutional neural networks},
  description={
  }
}
\newglossaryentry{cnn}{
  type=\acronymtype,
  name={CNN},
  description={Convolutional Neural Network},
  first={Convolutional Neural Network (CNN)\glsadd{cnng}},
  plural={CNNs},
  see=[Glossary:]{cnng}
}

% computational differentiation
\newglossaryentry{computational differentiation}{
  name={computational differentiation},
  description={
    Another term for \gls{adg}.
  }
}

% complex cell
\newglossaryentry{complex cell}{
  name={complex cell},
  plural={complex cells},
  description={
  }
}

% computation graph
\newglossaryentry{computation graph}{
  name={computation graph},
  description={
  }
}

% condition number
\newglossaryentry{condition number}{
  name={condition number},
  description={
    See \gls{Hessian condition number}
  }
}

% conjugate gradients
\newglossaryentry{conjugate gradients}{
  name={conjugate gradients},
  description={
    An approximate \gls{second-order method}
  }
}

% convolution
\newglossaryentry{convolution}{
  name={convolution},
  description={
  }
}

% convolutional layer
\newglossaryentry{convolutional layer}{
  name={convolutional layer},
  plural={convolutional layers},
  description={
  }
}

% cost function
\newglossaryentry{cost function}{
  name={cost function},
  plural={cost functions},
  description={ 
  }
}

% criterion
\newglossaryentry{criterion}{
  name={criterion},
  description={
  }
}

% critical point
\newglossaryentry{critical point}{
  name={critical point},
  plural={critical points},
  description={
    A point where the derivative of a function becomes 0
  }
}

% CPU
\newglossaryentry{cpug}{
  name={CPU},
  description={
    A specialised electronic circuit 
    which controls the execution of instructions
    in a computer system.
  }
}
\newglossaryentry{cpu}{
  type=\acronymtype,
  name={CPU},
  description={Central Processing Unit},
  first={Central Processing Unit (CPU)\glsadd{cpug}},
  plural={CPUs},
  firstplural={Central Processing Units (CPUs)},
  see=[Glossary:]{cpug}
}

% cross-correlation
\newglossaryentry{cross-correlation}{
  name={cross-correlation},
  description={
  }
}

