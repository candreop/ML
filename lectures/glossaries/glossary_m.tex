%
% Glossary / Entries starting from M
%
% Prof. Costas Andreopoulos <c.andreopoulos@cern.ch>
% ~~~~~~~~~~~~~~~~~~~~~~~~~~~~~~~~~~~~~~~~~~~~~~~~~~~~~~~~~~~~~~~~~~~~~~~~~~~~~
%

% Mark 1 perceptron
\newglossaryentry{Mark 1 perceptron}{
  name=Mark 1 perceptron,
  description={
    The original, custom-built hardware implementation of perceptron
  }
}

% maximum likelihood estimation
\newglossaryentry{maximum likelihood estimation}{
  name={maximum likelihood estimation},
  description={
  }
}

% maximum likelihood estimator
\newglossaryentry{maximum likelihood estimator}{
  name={maximum likelihood estimator},
  description={
    A \gls{point estimator} derived using \gls{maximum likelihood estimation} 
  }
}

% max pooling
\newglossaryentry{max pooling}{
  name={max pooling},
  description={
  }
}

% McCulloch-Pitts neuron
\newglossaryentry{McCulloch-Pitts neuron}{
  name=McCulloch-Pitts neuron,
  description={
    A single-layer perceptron
  }
}

% mean squared error loss function
\newglossaryentry{mean squared error loss function}{
  name={\gls{mse} loss function},
  description={
  }
}

% mini batch gradient descent
\newglossaryentry{mini batch gradient descent}{
  name={mini batch gradient descent},
  %hyphenated={},
  description={
  }
}

% ML
\newglossaryentry{mlg}{
  name={machine learning},
  description={
  }
}
\newglossaryentry{ml}{
  type=\acronymtype,
  name={ML},
  description={Machine Learning},
  first={Machine Learning (ML)\glsadd{mlg}},
  see=[Glossary:]{mlg}
}

% % MNIST
% \newglossaryentry{mnistg}{
%   name={MNIST},
%   description={
%   }
% }
% \newglossaryentry{mnist}{
%   type=\acronymtype,
%   name={MNIST},
%   description={Modified National Institute of Standards and Technology},
%   first={Modified National Institute of Standards and Technology (MNIST)\glsadd{mnistg}},
%   see=[Glossary:]{mnistg}
% }

% momentum
\newglossaryentry{momentum}{
  name={momentum},
  description={
    An extension of the \gls{gradient descent} method,
    allowing an inertia in the direction of the search for 
    the optimal solution
  }
}

% momentum parameter
\newglossaryentry{momentum parameter}{
  name={momentum parameter},
  description={
    A \gls{hyperparameter} of \gls{momentum}-based learning algorithms
  }
}

% MSE
\newglossaryentry{mse}{
  type=\acronymtype,
  name={MSE},
  description={Mean Squared Error},
  first={Mean Squared Error (MSE)}
}

