%
% Glossary / Entries starting from D
%
% Prof. Costas Andreopoulos <c.andreopoulos@cern.ch>
% ~~~~~~~~~~~~~~~~~~~~~~~~~~~~~~~~~~~~~~~~~~~~~~~~~~~~~~~~~~~~~~~~~~~~~~~~~~~~~
%

% data-generating distribution
\newglossaryentry{data-generating distribution}{
  name={data-generating distribution},
  plural={data-generating distributions},
  description={
  }
}

% denoising
\newglossaryentry{denoising}{
  name={denoising},
  description={
    A \gls{ml} task whereby a model is
    presented with a corrupted data example 
    $\vect{\tilde{x}} \in \mathbb{R}^n$,
    produced by some unknown corrupting process acting 
    $\vect{x} \in \mathbb{R}^n$, 
    and it is asked to predict
    the original example $\vect{x}$ or the
    conditional probability $p(\vect{x}|\vect{\tilde{x}})$
  }
}

% density estimation
\newglossaryentry{density estimation}{
  name={density estimation},
  description={
    A \gls{ml} task where model is asked to
    learn a function $p_{model}: \mathbb{R}^n \rightarrow \mathbb{R}$,
    and where $p_{model}(\vect{x})$ can be interpreted as a 
    \gls{probability density function} (\gls{probability mass function}) 
    if $\vect{x}$ is continuous (discrete)
  }
}

% derivative
\newglossaryentry{derivative}{
  name={derivative},
  plural={derivatives},
  description={
  }
}

% deep ai
\newglossaryentry{deep ai}{
  name={deep AI},
  description={
  Another term for artificial general intelligence (\gls{agi})
  }
}

% delta-bar-delta
\newglossaryentry{delta-bar-delta}{
  name={delta-bar-delta},
  description={
    An \gls{adaptive subgradient} algorithm
  }
}

% % design matrix
\newglossaryentry{design matrix}{
  name={design matrix},
  description={
    A design matrix is common way of representing small datasets.
    It is a matrix that contains a single data example in each row
  }
}

% dimensionality reduction
\newglossaryentry{dimensionality reduction}{
  name={dimensionality reduction},
  description={
    The task of reducing the number of features in a dataset 
  }
}

% DL
\newglossaryentry{dlg}{
  name={deep learning},
  description={
  }
}
\newglossaryentry{dl}{
  type=\acronymtype,
  name={DL},
  description={Deep Learning},
  first={Deep Learning (DL)\glsadd{dlg}},
  see=[Glossary:]{dlg}
}

