%
% Glossary / Entries starting from R
%
% Prof. Costas Andreopoulos <c.andreopoulos@cern.ch>
% ~~~~~~~~~~~~~~~~~~~~~~~~~~~~~~~~~~~~~~~~~~~~~~~~~~~~~~~~~~~~~~~~~~~~~~~~~~~~~
%


% ReLU
\newglossaryentry{relug}{
  name={rectified linear unit},
  plural={rectified linear units},
  description={
    A type of simple and piecewise linear \gls{activation function} 
    (0 for negative values of its argument, x, and equal to x otherwise) that,
    nevertheless, introduces the property of non-linearity to a \gls{dlg} model.
  }
}
\newglossaryentry{relu}{
  type=\acronymtype,
  name={ReLU},
  description={Rectified Linear Unit},
  first={Rectified Linear Unit (ReLU)\glsadd{relug}},
  plural={ReLUs},
  see=[Glossary:]{relug}
}

% reactive ai
\newglossaryentry{reactive ai}{
  name={reactive AI},
  description={
    Task-specific \gls{ai} system that reacts to inputs in predictable manner,
    but does not store memories and does not learn from experience
  }
}

% regression
\newglossaryentry{regression}{
  name={regression},
  description={
    A technique that relates a dependent variable 
    to one or more independent variables
  }
}

% regularisation
\newglossaryentry{regularisation}{
  name={regularisation},
  description={
    A technique giving a learning algorithm preference 
    for one solution over another in its hypothesis space,
    to help solve the problem of overfitting
  }
}

% regulariser
\newglossaryentry{regulariser}{
  name={regulariser},
  description={
    A penalty added to the cost function 
  }
}

% representational capacity
\newglossaryentry{representational capacity}{
  name={representational capacity},
  description={
    The family of functions a \gls{ml} algorithm 
    can choose from to reduce its training error
  }
}

% ResNet
\newglossaryentry{ResNet}{
  name={ResNet},
  description={
    A \glspl{cnn} architecture.
    Winner of \gls{ilsvrc} for 2015.
  }
}

% RMSprop
\newglossaryentry{RMSPropg}{
  name={Root Mean Square Propagation},
  description={
    An \gls{adaptive subgradient} algorithm
  }
}
\newglossaryentry{RMSProp}{
  type=\acronymtype,
  name={RMSProp},
  description={Root Mean Square Propagation},
  first={Root Mean Square Propagation (RMSProp)\glsadd{RMSPropg}},
  see=[Glossary:]{RMSPropg}
}
