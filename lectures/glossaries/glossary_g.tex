%
% Glossary / Entries starting from G
%
% Prof. Costas Andreopoulos <c.andreopoulos@cern.ch>
% ~~~~~~~~~~~~~~~~~~~~~~~~~~~~~~~~~~~~~~~~~~~~~~~~~~~~~~~~~~~~~~~~~~~~~~~~~~~~~
%

% GAN
\newglossaryentry{gang}{
  name={generative adversarial network},
  description={
  }
}
\newglossaryentry{gan}{
  type=\acronymtype,
  name={GAN},
  description={Generative Adversarial Network},
  first={Generative Adversarial Network (GAN)\glsadd{gang}},
  plural={GANs},
  firstplural={Generative Adversarial Networks (GANs)},
  see=[Glossary:]{gang}
}

% generalisation
\newglossaryentry{generalisation}{
  name={generalisation},
  description={
    The property of a \gls{ml} model performing well for previously unseen inputs
  }
}

% generalisation error
\newglossaryentry{generalisation error}{
  name={generalisation error},
  description={
    Another term for \gls{test error}
  }
}

% global minimum
\newglossaryentry{global minimum}{
  name={global minimum},
  plural={global minima},
  description={
    A \gls{critical point}, 
    where a function takes lower values 
    than in all points in the domain of a function
  }
}

% global maximum
\newglossaryentry{global maximum}{
  name={global maximum},
  plural={global maxima},
  description={
    A \gls{critical point}, 
    where a function takes higher values 
    than in all points in the domain of a function
  }
}

% GoogLeNet
\newglossaryentry{GoogLeNet}{
  name={GoogLeNet},
  description={
    A \glspl{cnn} architecture.
    Winner of \gls{ilsvrc} for 2014.
    Also known as \gls{Inception} (v1)
  }
}

% GPU
\newglossaryentry{gpug}{
  name={GPU},
  description={
    A specialised electronic circuit with a parallel structure that
    makes them more efficient than \glspl{CPU} for processing large blocks of data in parallel
  }
}
\newglossaryentry{gpu}{
  type=\acronymtype,
  name={GPU},
  description={Graphics Processing Unit},
  first={Graphics Processing Unit (GPU)\glsadd{gpug}},
  plural={GPUs},
  firstplural={Graphics Processing Units (GPUs)},
  see=[Glossary:]{gpug}
}

% gradient 
\newglossaryentry{gradient}{
  name={gradient},
  plural={gradients},
  description={
  }
}

% gradient descent
\newglossaryentry{gradient descent}{
  name={gradient descent},
  hyphenated={gradient-descent},
  description={
  }
}
