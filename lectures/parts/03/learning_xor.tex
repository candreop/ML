%
% Learning the XOR function
%

\begin{frame}[t,allowframebreaks]{Learning the XOR function -} 

    The XOR (exclusive or, or exclusive disjunction) function
    is an operation on two variables that take binary values (0,1).\\
    \vspace{0.2cm}

    \begin{columns}[t]
        \begin{column}{0.32\textwidth}
            \vspace{-1.2cm}
            \begin{center}
                \begin{tabular}{ c c | c }
                 $x_1$ & $x_2$ & $y = x_1 \oplus x_2$ \\ 
                 \hline
                 0 & 0 & 0 \\  
                 0 & 1 & 1 \\   
                 1 & 0 & 1 \\  
                 1 & 1 & 0 \\   
                \end{tabular}
                %\label{tab:xor_truth}
            \end{center}
        \end{column}
        \begin{column}{0.68\textwidth}
            Let $x_1$, $x_2$ be the two input binary values
            and $y$ be the output of the operation, denoted as $x_1 \oplus x_2$.\\
            \vspace{0.2cm}
            $y$ is 1 only if exactly one of $x_1$, $x_2$ is 1.\\ 
            \vspace{0.2cm}
            The full truth table is given on the left.\\
        \end{column}
    \end{columns}

    \vspace{0.4cm}
    We would like to construct a model 
    $f(\mathbf{x};\mathbf{\theta}):\{0,1\}^2\rightarrow\{0,1\}$ 
    that {\bf learns the XOR function} 
    by adjusting the parameters in $\mathbf{\theta}$.\\

    \begin{itemize}
        \item 
        This function is defined only for 
        4 points $\mathbf{x}=(x_1,x_2)^T$.
        \begin{itemize}
         \item 
         $\mathbf{x} \in \mathbb{X} 
          = \{(0,0)^T, (0,1)^T, (1,0)^T, (1,1)^T\}$.
        \end{itemize}
        \item 
        The training set $\mathbb{D}$ contains 
        4 training examples $(\mathbf{x},y)$,
        constructed from $\mathbf{x} \in \mathbb{X}$ 
        and the truth table given above.
    \end{itemize}

\framebreak

    Let $\hat{y}=f(\mathbf{x};\mathbf{\theta})$ be our prediction of $y$ for a 
    given value of $\mathbf{x}=(x_1,x_2)$.\\
    \vspace{0.3cm}
    We can treat this as a regression problem, and use a 
    \index{mean squared error} \index{loss function}
    \gls{mean squared error loss function}:
    \begin{equation}
        \mathcal{L}(\mathbf{\theta}) =  
        \frac{1}{4} 
        \sum_{(\mathbf{x},y) \in \mathbb{D}} 
        (y - \hat{y})^2 =
        \frac{1}{4} 
        \sum_{(\mathbf{x},y) \in \mathbb{D}} 
        (y - f(\mathbf{x};\mathbf{\theta}))^2
        \label{eq:learn_xor_loss_function_1}
    \end{equation}
    This is not an ideal choice for binary data,
    but it is a simple choice which is sufficient for our purposes.\\
    \vspace{0.2cm}

    What remains, is to specify the form of the model 
    $f(\mathbf{x};\mathbf{\theta})$.

\end{frame}
