
\begin{frame}[t]{Point estimation}

    In statistics, 
    \index{point estimation}\gls{point estimation}
    is the {\bf derivation of a best estimate
    for a quantity of interest}, using a sample of data.
    \begin{itemize}
        \small
        \item
        Typically, the quantity of interest is a 
        single parameter, or a vector of parameters, 
        controlling a parametric model.
        \item
        To distinguish the estimates of parameters of
        interest from their true or other values,
        we will use $\vect{\hat{\theta}}$ to denote the 
        point estimate of $\vect{\theta}$.
    \end{itemize}

    \vspace{0.2cm}

    If $\mathbb{X}_{m}$ = $\{\vect{x}_{1}, ..., \vect{x}_{m}\}$ is a set
    of $m$ independent and identically distributed (i.i.d.) points,
    a \index{point estimator}\gls{point estimator}\footnote{
        also known as a {\em statistic}} 
    is any function of the data:\\
    \vspace{-0.1cm}
    \begin{equation}
        \vect{\hat{\theta}}_{\mathbb{X}_{m}} = g(\vect{x}_{1}, ..., \vect{x}_{m})
        \label{eq:point_estimator}
    \end{equation}\\

    \vspace{0.2cm}
    While any function qualifies as a \gls{point estimator}, 
    a good estimator gets close to the true value of $\vect{\theta}$
    that generated the sample of data.\\

    \vspace{0.2cm}
    Since the data are drawn randomly, 
    $\vect{\hat{\theta}}_{\mathbb{X}_{m}}$ itself is a random variable.\\

\end{frame}