%
%
%

\begin{frame}[t]{Learning by experience}

    A \index{ML}\gls{ml} algorithm can {\bf learn from experience}.\\
    \vspace{0.2cm}
    Often, this should be understood as {\bf experiencing a dataset}.
    \begin{itemize}
        \item 
        \small
        A dataset is a collection of several data examples or data ``points".\\
    \end{itemize}
    \vspace{0.1cm}
    Depending on the type of experience,
    we distinguish between:\\
    \vspace{0.1cm}
    \begin{itemize}
        \item 
        {\bf \index{supervised learning}\gls{supervised learning}}, 
        where the data examples include {\bf labels} to teach the algorithm
        how to make correct decisions,
        and\\
        \vspace{0.1cm}
        \item 
        {\bf \index{unsupervised learning}\gls{unsupervised learning}},
        where the algorithm is trying to deduce relationships and patterns
        within a set of {\bf unlabelled} examples.\\
    \end{itemize}
    \vspace{0.1cm}
    The above {\bf distinction is often blurred}.\\

    \vspace{0.2cm}

    Some \gls{ml} algorithms do not experience a fixed dataset:
    Instead, they are allowed to {\bf interact with the environment}.
    \begin{itemize}
        \item 
        The term
        {\bf \index{reinforcement learning}\gls{reinforcement learning}}
        describes this learning paradigm.\\
    \end{itemize}

\end{frame}

