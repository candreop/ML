
%
%
%

\begin{frame}{Types of Artifical Intelligence}

    Classification based on capabilities (Type-1):\\
    \vspace{0.1cm}

    \begin{itemize}

        \item 
        \index{artificial narrow intelligence}\index{ANI}
        \gls{ani}, often referred to as 
        \index{weak AI}\Gls{weak ai} or
        \index{narrow AI}\Gls{narrow ai}:
        This is \index{artificial intelligence}\index{AI}\gls{ai} 
        with a {\bf narrow range of capabilities}.\\

        \vspace{0.1cm}

        \begin{itemize}
            \item 
            Only type of \gls{ai} that is {\bf successfully implemented to date}
            \item
            Examples: 
            self-driven cars, voice assistants, 
            face recognition tools, spam filters,
            news feed personalization in social media, 
            chatbots etc.
            \item 
            Trained for specific tasks in a {\bf narrow domain}.   
            \item 
            {\bf Imitates intelligence}, rather than being intelligent.\\
        \end{itemize}

        \vspace{0.1cm}

        \item 
        \index{artificial general intelligence}\index{AGI}
        \gls{agi}, often referred to as 
        \index{strong AI}\Gls{strong ai} or
        \index{deep AI}\Gls{deep ai}:
        \gls{ai} with {\bf human-level capabilities}.\\

        \vspace{0.1cm}

        \begin{itemize}
            \item 
            {\bf Autonomous systems} that {\bf can learn any task}.
            \item 
            Currently, there is no such system.
            \item
            72 \index{agi} R\&D projects were identified as being active in 2020 \cite{GCRI:2020agi}.
        \end{itemize}


        \vspace{0.1cm}
        \item 
        \index{artificial superintelligence}\index{ASI} 
        \gls{asi}:
        \gls{ai} that greatly {\bf exceeds human capabilities} - A hypothetical concept.

    \end{itemize}
    
\end{frame}

%

\begin{frame}{Types of Artifical Intelligence}

    Classification based on functionality (Type-2):\\

    \begin{itemize}

        \item 
        \index{reactive AI}\index{artificial intelligence}\index{AI}\Gls{reactive ai}
        \begin{itemize}
            \item 
            Task-specific \gls{ai} that reacts to inputs and is 
            predictable (it always produces a given response for a given set of inputs)
            \item 
            Does not store memories and does not learn from experience.
            \item
            Example: IBM's Deep Blue chess-playing \gls{ai} system
        \end{itemize}

        \item 
        \index{limited memory AI}\Gls{limited memory ai}
        \begin{itemize}
            \item 
            Task-specific \gls{ai} that stores memories (for a limited time period) 
            and learns from its experiences.
            \item
            Example: Self-driven cars
        \end{itemize}

        \item 
        \index{theory of mind AI}\Gls{theory of mind ai}
        \begin{itemize}
            \item 
            Next-level \gls{ai} system, able to learn any task and exhibit common sense.
            \item 
            Emotionally intelligent, would be able to interact with human emotions
            and adjust its behaviour, show empathy, and understand moral norms.
        \end{itemize}

        \item 
        \index{self-awareness AI}\Gls{self-awareness ai}
        \begin{itemize}
            \item Would attribute mental states not only to others, but also to self.
            \item \gls{ai} with human-level artificial consciousness.
        \end{itemize}

    \end{itemize}
    
\end{frame}
