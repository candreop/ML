\renewcommand{\prevpart}{0 }
\renewcommand{\thispart}{1 }
\renewcommand{\nextpart}{2 }

\section{Introduction to Machine Learning}

% Cover page
%
% Cover page for giveb part
%

\title[\modulename - Part \thispart]
{
  {\bf 
   \modulename - 
   Part \thispart\\
  }
  \vspace{0.5cm}
  {\it 
   \color{yellow}
    \secname\\
  }
}
\author[C.Andreopoulos] {
  Professor Costas Andreopoulos\inst{1,2}, {\it FHEA}
}
\institute[Liverpool/STFC-RAL] {
   \inst{1} University of Liverpool, Department of Physics\\
   \vspace{0.1cm}
   \inst{2} U.K. Research \& Innovation (UKRI), Science \& Technology Facilities Council,\\
            Rutherford Appleton Laboratory, Particle Physics Department\\
   \vspace{0.5cm}
   {\it {\color{magenta} Lectures delivered at the University of Liverpool, 2023-24}}\\
   \vspace{0.2cm}
}
\date{\today}

\titlegraphic{
  \includegraphics[height=25px]{lectures/img/logo/liverpool.png}
  \hspace{3px}
  \includegraphics[height=30px]{lectures/img/logo/ral.png}
}

\begin{frame}[plain]
  \titlepage
\end{frame}




% Outline
%
% Table of contents to be displayed at the beginning of each part
%

\begin{frame}[t,allowframebreaks]{Outline for Part \thispart -}
  % Part \thispart (\secname) covers the following topics:\\
  % \vspace{0.5cm}
  \linespread{1.1}
  \setcounter{secnumdepth}{3}
  \setcounter{tocdepth}{3}
  % \tableofcontents[currentsection, hideothersubsections, sectionstyle=hide/hide]
  \tableofcontents[part=\thispart]
\end{frame}



% Precursors
\subsection{Precursors of artificial intelligence}
%
%
%

\begin{frame}[t, allowframebreaks]{Precursors of Artifial Intelligence -}

    \vspace{-0.1cm}
    We have always dreamt of building machines with human-like intelligence!\\
    \begin{itemize}
        \small
        \item
        We have myths and stories of master craftsmen and intelligent machines.\\
    \end{itemize}
    \vspace{-0.4cm}

    \begin{columns}[t]
        \begin{column}{0.46\textwidth}
         \begin{center}
          \includegraphics[width=0.99\textwidth]{./images/precursors/talos.jpg}\\
          {\scriptsize 
          \vspace{0.1cm}
          Talos on a Greek coin from 300 BCE.\\
          \color{col:attribution} 
          (Cabinet des Médailles, Paris.)
          \href{https://en.wikipedia.org/wiki/Talos\#/media/File:Didrachm_Phaistos_obverse_CdM.jpg}{\tiny [link]}
          \\}
         \end{center}
        \end{column}
        \begin{column}{0.54\textwidth}
            \begin{itemize}
                \small
                \item 
                {\bf Talos}, the bronze giant built by the god
                Hephaestus to protect Crete from invaders.
                \href{https://en.wikipedia.org/wiki/Talos}{\tiny [Wikipedia]}
                \item 
                {\bf Galatea}, the female ivory statue animated 
                by the goddess Aphrodite when its sculptor, Pygmalion, 
                fell in love with it.
                \href{https://en.wikipedia.org/wiki/Pygmalion_(mythology)}{\tiny [Wikipedia]}
                \item
                {\bf Golems} in Jewish folklore, made from clay 
                and animated by words written on their foreheads, 
                or on pieces of paper placed in their mouth.
                \href{https://en.wikipedia.org/wiki/Golem}{\tiny [Wikipedia]}
                \item
                {\bf Homunculi} (sing.: Homunculus), 
                the little anthropomorphic creatures found in alchemical traditions.
                \href{https://en.wikipedia.org/wiki/Homunculus}{\tiny [Wikipedia]}
            \end{itemize}        
        \end{column}
    \end{columns}

    \framebreak

    \begin{itemize}
        \small
        \item
        {\bf Brazen heads}, the future-telling automata of 
        Gerbert of Aurillac, Saint Albertus, Robert Grosseteste, 
        and Roger Bacon.
        \href{https://en.wikipedia.org/wiki/Brazen_head}{\tiny [Wikipedia]}\\
        \item
        {\bf Frankenstein!} by Marry Shelley (1818).
        \includegraphics[width=0.04\textwidth]{./images/precursors/stein.jpg}
        \href{https://en.wikipedia.org/wiki/Frankenstein}{\tiny [Wikipedia]}\\
        \item
        {\bf Hadaly}, the female android in the science fiction novel `The Future Eve' (1886)
        by Auguste Villiers de I'Isle-Adam.
        \href{https://en.wikipedia.org/wiki/The_Future_Eve}{\tiny [Wikipedia]}\\
        \item
        {\bf Maria} in the science fiction movie `Metropolis' (1927)
        \href{https://en.wikipedia.org/wiki/Metropolis_(1927_film)}{\tiny [Wikipedia]}\\
    \end{itemize}        

    \vspace{-0.3cm}

    \begin{columns}[t]
        \begin{column}{0.32\textwidth}
            \begin{center}
                \includegraphics[width=0.99\textwidth]
                {./images/precursors/homunculus_cropped.png}\\
                {\scriptsize 
                \vspace{0.1cm}
                19$^{th}$ century engraving of Wagner and Homunculus 
                from Goethe's Faust II.
                \href{https://en.wikipedia.org/wiki/Homunculus\#/media/File:Faust_image_19thcentury.jpg}{\tiny [link]}\\
                }
            \end{center}
        \end{column}
        \begin{column}{0.37\textwidth}
            \begin{center}
                \includegraphics[width=0.99\textwidth]
                {./images/precursors/brazen_head_cropped.png}\\
                {\scriptsize 
                \vspace{0.1cm}
                Reprint of 1630 edition of 
                Robert Greene's Friar Bacon and Friar Bungay.
                \href{https://en.wikipedia.org/wiki/Friar_Bacon_and_Friar_Bungay\#/media/File:Greene_Bacon_and_Bungay_1630.jpg}{\tiny [link]}\\
                }
            \end{center}
        \end{column}
        \begin{column}{0.30\textwidth}
            \begin{center}
                \includegraphics[width=0.80\textwidth]
                {./images/precursors/metropolis_cropped.png}\\
                {\scriptsize 
                \vspace{0.1cm}
                Part of the theatrical release poster for Metropolis,
                by Heinz Schulz-Neudamn.
                \href{https://en.wikipedia.org/wiki/Metropolis_(1927_film)\#/media/File:Metropolis_(German_three-sheet_poster).jpg}{\tiny [link]}\\
                }      
            \end{center}
        \end{column}
    \end{columns}

\end{frame}

% History of AI
\subsection{History of artificial intelligence}
\input{parts/01/ai_history_eras.tex}
\subsubsection{Birth of AI (1952-1956)}
%
%
%

\begin{frame}[t]{Birth of AI (1952-1956)} 

    Several developments in the first half of the 20th century:
    \begin{itemize}
        \item Discovery that the {\bf brain is an electrical network of neurons}
        \begin{itemize}
            \item 
              Richard Caton reported to British Medical Association 
              the first observation of electrical impulses from the 
              brains of living animals (1875) \cite{Caton:1875}.
            \item 
              Hans Berger recorded the first human 
              electroencephalogram (1924) \cite{Berger:1929}.
        \end{itemize}
        \item 
        Development of {\bf control and communication theory} (cyberneutics) 
        \begin{itemize}
            \item by Norbert Wiener
        \end{itemize}
        \item 
        Development of {\bf information theory} 
        \href{https://en.wikipedia.org/wiki/Information_theory}{\tiny [Wikipedia]}
        \begin{itemize}
            \item Early work by Harry Nyquist and Ralph Hartley in 1920's.
            \item Foundations established by Claude Shannon in the 1940's \cite{Shannon:1948}.
        \end{itemize}
        \item 
        Development of {\bf theory of computation} 
        \href{https://en.wikipedia.org/wiki/Theory_of_computation}{\tiny [Wikipedia]}
        \begin{itemize}
            \item John von Neumann, Alan Turing, Alonzo Church and others
        \end{itemize}
    \end{itemize}
    \vspace{0.2cm}
    In early 1950's it became possible to {\bf start imagining a digital brain}!
            
\end{frame}
    
    
    


\subsubsection{Symbolic AI (1956-1974)}
\subsubsection{First AI winter (1974-1980)}
\subsubsection{Boom (1980-1987)}
\subsubsection{Bust: Second AI winter (1987-1993)}
\subsubsection{AI (1993-2011)}
\subsubsection{Deep learning, big data and AI (2011-present)}


%
%
%

\begin{frame}[t]{Effect of increased data}

    \begin{columns}
        \begin{column}{0.50\textwidth}
         \begin{center}
          \includegraphics[width=0.95\textwidth]{./images/dl_intro/accuracy_vs_amount_of_data_1.png}\\
          {\scriptsize \color{col:attribution} 
          Image reproduced from p.54 of \cite{Aggarwal:2018SpringerDL}}\\
         \end{center}
        \end{column}
        \begin{column}{0.50\textwidth}
        \end{column}
    \end{columns}


\end{frame}

%
%
%

\begin{frame}[t,allowframebreaks]{Increasing neural network size - }

    % Intro

    % Number of connections in various artificial neural nets as a function of time
    % and comparison with biological brains

    The human brain has $\sim$100 billion neurons and $\sim$100 trillion synapses!

    \begin{center}
        \includegraphics[width=0.95\textwidth]
          {./images/dl_intro/nnet_size_connections_vs_time_01.png}\\
        {\scriptsize \color{col:attribution} 
        Reproduced from p.22 of \cite{Goodfellow:2017DL}}\\
    \end{center}

    \framebreak

    % Number of neurons in various artificial neural nets as a function of time
    % and comparison with biological brains

    \begin{center}
        \includegraphics[width=0.95\textwidth]
           {./images/dl_intro/nnet_size_neurons_vs_time_01.png}\\
        {\scriptsize \color{col:attribution} 
        Reproduced from p.23 of \cite{Goodfellow:2017DL}}\\
    \end{center}
       {\tiny
       1. Perceptron (1958) \cite{Rosenblatt:1958p},
       2. Adaptive linear element (1960) \cite{Widrow:1960as},
       3. Neocognitron (1980) \cite{Fukushima:1980nc},
       4. Early back-propagation network (1986) \cite{Rumelhart:1986erp},
       5. Recurrent neural network for speech recognition (1991) \cite{Robinson:1991rerp},
       6. Multilayer perceptron for speech recognition (1991) \cite{Bengio:1991pma},
       7. Mean field sigmoid belief network (1996) \cite{Saul:1996mf},
       8. LeNet-5 (1998) \cite{LeCun:1998ln5},

       9. Echo state network (2004) (Jaeger and Haas, 2004)
       10. Deep belief network (2006) (Hinton et al., 2006)
       11. GPU-accelerated convolutional network (2006) (Chellapilla et al., 2006)
       12. Deep Boltzmann machine (2009) (Salakhutdinov and Hinton, 2009a)
       13. GPU-accelerated deep belief network (2009) (Raina et al., 2009)
       14. Unsupervised convolutional network (2009) (Jarrett et al., 2009)
       15. GPU-accelerated multilayer perceptron (2010) (Ciresan et al., 2010)
       16. OMP-1 network (2011) (Coates and Ng, 2011)
       
       17. Distributed autoencoder (2012) \cite{Le:2012daut}
       18. Multi-GPU convolutional network (2012) \cite{Krizhevsky:2012img},
       19. COTS HPC unsupervised convolutional network (2013) \cite{Coates:2013cots},       
       20. GoogLeNet (2014) \cite{Szegedy:2014gnet}\\
       }

    \framebreak

    % Information from recent well-known artificial neural networks

    \begin{itemize}
        \item GPT-2 had 1.5 billion parameters and around 50 billion neurons
        \item GPT-3 is estimated to have around 60-80 billion neurons
        \item GPT-4 is estimated to have around 60-80 billion neurons
    \end{itemize}

\end{frame}


\subsection{Human vs computer learning}
%
%
%


\begin{frame}[t,allowframebreaks]{Human vs computer intelligence -} 

\begin{itemize}
\item Computers can 
\end{itemize}

\end{frame}

\subsection{Different learning paradigms: Supervised, unsupervised and reinforcement learning}

\subsection{Artificial intelligence, machine learning and deep learning}

\subsection{Machine learning tasks}

\subsection{A simple practical example: Linear regression}

\subsection{Biologically inspired methods of computer learning}

\subsection{Basic architecture of neural networks}

\subsection{Fundamental concepts}
\begin{frame}[t,allowframebreaks]{Fundamental concepts - }

    \gls{relu}
\end{frame}


% From AI to DL
\begin{frame}{Connection between \acrshort{ai}, \acrshort{ml} and \acrshort{dl}}

    \begin{center}
        \includegraphics[width=0.90\textwidth]{./images/dl_intro/ai_ml_dl.png}\\
        {\scriptsize Image reproduced from \cite{NVidiaBlog:DifferenceBetweenAIMLDL}}\\
    \end{center}

\end{frame}
    


% Main points to remember
\renewcommand{\partsummarytitle}{Main points to remember }
\renewcommand{\summarizedpart}{1 }

%
%
%

\begin{frame}{Part \summarizedpart - \partsummarytitle}


\end{frame}



% Preview of next part
\begin{frame}{Preview of Lecture \nextlecture}

\begin{itemize}
{\small
\item blah
\item blah
}
\end{itemize}

\end{frame}



% Suggested reading for this part
\subsection{Suggested reading}
%
%
%

\begin{frame}{Suggested reading for Part \thispart}

{
\small

}
    

\end{frame}


% Optional material


