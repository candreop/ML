%
%
%

\begin{frame}[t]{Birth of AI (1952-1956)} 

    Several developments in the first half of the 20th century:
    \begin{itemize}
        \item Discovery that the {\bf brain is an electrical network of neurons}
        \begin{itemize}
            \item 
              Richard Caton reported to British Medical Association 
              the first observation of electrical impulses from the 
              brains of living animals (1875) \cite{Caton:1875}.
            \item 
              Hans Berger recorded the first human 
              electroencephalogram (1924) \cite{Berger:1929}.
        \end{itemize}
        \item 
        Development of {\bf control and communication theory} (cyberneutics) 
        \begin{itemize}
            \item by Norbert Wiener
        \end{itemize}
        \item 
        Development of {\bf information theory} 
        \href{https://en.wikipedia.org/wiki/Information_theory}{\tiny [Wikipedia]}
        \begin{itemize}
            \item Early work by Harry Nyquist and Ralph Hartley in 1920's.
            \item Foundations established by Claude Shannon in the 1940's \cite{Shannon:1948}.
        \end{itemize}
        \item 
        Development of {\bf theory of computation} 
        \href{https://en.wikipedia.org/wiki/Theory_of_computation}{\tiny [Wikipedia]}
        \begin{itemize}
            \item John von Neumann, Alan Turing, Alonzo Church and others
        \end{itemize}
    \end{itemize}
    \vspace{0.2cm}
    In early 1950's it became possible to {\bf start imagining a digital brain}!
            
\end{frame}
    
    
    