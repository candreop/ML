%
% The Computation Graph
%

\begin{frame}[t]{The Computation Graph} 

    Any function that can be written down algebraically, 
    can be {\bf decomposed to elementary functions and operations}.\\
    \vspace{0.2cm}
    This {\bf decomposition can be organised in a \gls{computation graph}},
    which allows us to understand the structure of the function and
    to evaluate it in a programmatic way.\\
    \vspace{0.2cm}
    
    \begin{blockexample}{Graphs}
    {\small
    A graph is a network of points (also called graph vertices, or nodes)
    and lines (also called graph edges, or arcs) connecting some subset of points.
    
    There are several types of graphs.
    Here, we will work with simple, labeled, directional graphs:
    \begin{itemize}
        \scriptsize
        \item 
        A simple graph is one where any two points are
        connected, at most, by one line. 
        \item
        A labeled graph is one in which points, lines, or both, are assigned  labels so 
        the carry more information than what is encoded in their intrinsic connectivity.
        \item
        A directed graph is one in which lines have a direction (from a parent to a child node).
    \end{itemize}
    }
    \end{blockexample}
    
\end{frame}
