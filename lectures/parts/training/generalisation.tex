

\begin{frame}[t,allowframebreaks]{Generalisation -}

    \label{sec:Generalisation}

    Training a \gls{ml} model {\em looks like} an optimization task.\\
    \vspace{0.1cm}
    \begin{blockexample}{}
      Typically, to train a \gls{ml} model: 
      \begin{itemize}
        \item 
        We use a \index{training set}\gls{training set} 
        that contains a number of examples, and
        \item
        define a \index{loss function}\gls{loss function},
        a measure of training error over that set.
      \end{itemize}
      The training process adjusts the model parameters
      to optimize the model performance (minimise the training error).\\
    \end{blockexample}
    \vspace{0.2cm}
    However, {\bf training a \gls{ml} model is not a pure optimization problem}.\\
    \vspace{0.2cm}
    Our \gls{ml} model should {\bf maintain a satisfactory performance 
    with new, previously unseen inputs} - 
    not just for the inputs in the training set.\\
    \vspace{0.2cm}
    Performing well for previously unseen inputs is called 
    \index{generalisation}\gls{generalisation}.\\
    \vspace{0.2cm}
    How to build model that generalise is one of the central problems in \gls{ml}.


    \framebreak

    %
    %

    In the linear regression example, our model
    \begin{equation}
        \hat{y} = {\mathbf x}^T \cdot {\mathbf w}
        %\label{eq:}
    \end{equation}        

    was trained using a set $\mathbb{D}$ of $N_{(train)}$
    training examples $({\mathbf x_{(train)}},y_{(train)})$, 
    by minimizing the loss function
    \begin{equation}
        L_{(train)} = \frac{1}{N_{(train)}} 
        \sum_{({\mathbf x_{(train)}},y_{(train)})  \in \mathbb{D}}
        \Big( y_{(train)} - {\mathbf x_{(train)}}^T \cdot {\mathbf w} \Big)^2
        %\label{eq:}
    \end{equation}        

    However, we also care for the test (Generalisation) error
    \begin{equation}
        L_{(test)} = \frac{1}{N_{(test)}} 
        \sum_{({\mathbf x_{(test)}},y_{(test)})  \in \mathbb{T}}
        \Big( y_{(test)} - {\mathbf x_{(test)}}^T \cdot {\mathbf w} \Big)^2
        %\label{eq:}
    \end{equation}        
    for the $N_{(test)}$ test examples $({\mathbf x_{(test)}},y_{(test)})$ 
    of the set $\mathbb{T}$.

    \framebreak

    %
    %

    What can we say about the \gls{ml} model performance on the test set,
    if we only ever see the training set?
    \begin{itemize}
        \item
        Not much, unless the test and training data sets come from the 
        {\bf same data-generating process}.    
    \end{itemize}
    \vspace{0.2cm}   
    Typically, we make assumptions known as the {\bf i.i.d. assumptions}.\\
    \vspace{0.2cm}
    The examples in the test and training sets are assumed to be:
    \begin{itemize}
        \item {\bf independent}, and
        \item {\bf identically distributed}, 
        i.e. they are drawn from the same probability distribution 
        (the data-generating distribution)
    \end{itemize}
    \vspace{0.2cm}
    With the i.i.d assumptions:
    \begin{itemize}
        \item
        we can describe the data-generating process
        with a probability distribution over a single data example, and
        \item
        study the relationship between the training and test error.
    \end{itemize}

\end{frame}
