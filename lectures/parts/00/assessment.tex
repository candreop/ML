
\begin{frame}{Assessment}

\begin{itemize}

    \item 
    {\bf \color{red}SPECIFY}-hour open book examination on the {\bf \color{red}SPECIFY} exam period
    (Weight: {\bf \color{red}SPECIFY}\%)
    \vspace{0.2cm}

    \item
    Continuous Assessment
    (Weight: {\bf \color{red}SPECIFY}\%)

      \begin{itemize}
        \item  {\bf \color{red}SPECIFY DETAILS}
      \end{itemize}

\end{itemize}

\begin{block001}{Late penalties, deadline extensions, penalties for copying}
\begin{center}
{\tiny
  Late penalties should be applied as specified in the Code of Practice on Assessment (see Section 6.2 of the main document).\\
  The Code of Practice on Assessment does not provide for students to request extensions to coursework deadlines, 
  unless such extensions are allowed under a student’s Learning Support Plan.  
  Students submitting coursework late because of unforeseen medical or other extenuating circumstances may instead 
  apply for exemption from late penalties.\\
  Such exemption cannot be granted by individual Module Co-ordinators, but should be decided by Year Co-ordinators, 
  who will work within agreed guidelines to ensure consistency. \\
  Suspected cases of academic misconduct will be handled as described in Appendix L, in the Code of Practice on Assessment.\\
  \href{www.liverpool.ac.uk/media/livacuk/tqsd/code-of-practice-on-assessment/code\_of\_practice\_on\_assessment.pdf}
       {www.liverpool.ac.uk/media/livacuk/tqsd/code-of-practice-on-assessment/code\_of\_practice\_on\_assessment.pdf}\\
}
\end{center}   
\end{block001}
        
\end{frame}
