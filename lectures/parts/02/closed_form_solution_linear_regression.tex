%
% Give a closed form solution for the problem of linear regression.
%
% We are going to use the final result later to demonstrate that a linear 
% model can not learn the XOR function.
%

\begin{frame}[t,allowframebreaks]{Closed form solution for linear regression -} 

    Consider a linear function 
    $f: \mathbb{R}^{n} \rightarrow \mathbb{R}$,
    which is mapping a vector $\mathbf{x}$ of 
    $n$ real-valued components, $\mathbf{x}=(x_1,...x_{n})^T$,
    to a real-valued output, $\hat{y}$:\\
    \vspace{-0.2cm}
    \begin{equation}
        \hat{y} = f(\mathbf{x};\mathbf{w},b) = 
        \mathbf{x}^{T} \mathbf{w} + b =
        \sum_{j=1}^{n} x_j w_j + b 
        \label{eq:linear_regression_closed_form_solution_model_1}
    \end{equation}\\
    \vspace{-0.1cm}
    The weights $\mathbf{w}=(w_1,...w_{n})^T$ and bias $b$ are the 
    parameters of the model.\\
    \vspace{0.2cm}
    As usual, we can write Eq.~\ref{eq:linear_regression_closed_form_solution_model_1}
    in a more compact form, by
    introducing a new fixed input variable $x_0$ and
    considering the bias $b$ as the weight $w_0$ associated with $x_0$.
    Thus, the linear model can be rewritten as:\\
    \vspace{-0.2cm}
    \begin{equation}
        \hat{y} = f(\mathbf{x};\mathbf{w}) = 
        \mathbf{x}^{T} \mathbf{w} =
        \sum_{j=0}^{n} x_j w_j 
        \label{eq:linear_regression_closed_form_solution_model_2}
    \end{equation}\\
    \vspace{-0.1cm}
    where $\mathbf{x}=(x_0=+1, x_1,...x_{n})^T$ and $\mathbf{w}=(w_0=b, w_1,...w_{n})^T$.\\

    \framebreak

    The model shown in Eq.~\ref{eq:linear_regression_closed_form_solution_model_2}
    will need to learn its parameters $\mathbf{w}$ using a training set $\mathbb{D}$
    that includes several examples of $(\mathbf{x},y)$.\\
    \vspace{0.2cm}

    To quantify the model performance, we can use the
    \index{mean squared error} \index{loss function}
    \gls{mean squared error loss function}:
    \begin{equation}
        \mathcal{L}(\mathbf{w}) =  
        \frac{1}{N} 
        \sum_{(\mathbf{x},y) \in \mathbb{D}} 
        (y - \hat{y})^2 
        \label{eq:linear_regression_closed_form_solution_loss_1}
    \end{equation}

    If the training set $\mathbb{D}$ contains $N$ examples 
    $(\mathbf{x}^{(i)},y^{(i)})$, $i \in [1,N]$,
    Eq.~\ref{eq:linear_regression_closed_form_solution_loss_1}
    can be written as:
    \begin{equation}
        \mathcal{L}(\mathbf{w}) =  
        \frac{1}{N} 
        \sum_{i=1}^{N} 
        (y^{(i)} - \hat{y}^{(i)})^2 =
        \frac{1}{N} 
        \sum_{i=1}^{N} 
        \Big(y^{(i)} - \sum_{j=0}^{n} x_j^{(i)} w_j\Big)^2
        \label{eq:linear_regression_closed_form_solution_loss_2}
    \end{equation}

    \framebreak

    If a set of parameters $\mathbf{w}=\mathbf{w}^\star$ 
    is minimizing the \index{loss function}\gls{loss function} 
    of Eq.~\ref{eq:linear_regression_closed_form_solution_loss_2}, 
    then:
    \begin{equation}
        \frac{\partial \mathcal{L}(\mathbf{w})}{\partial w_k^\star} = 0
        \label{eq:linear_regression_closed_form_solution_zero_derivative}
    \end{equation}
    for each $k \in [0,n]$.
    Note that the converse is not true: 
    If $\partial \mathcal{L}(\mathbf{w})/\partial w_k^\star$ is zero, then
    $\mathbf{w}^\star$ could be either a minimum, a maximum or a saddle point.\\
    \vspace{0.2cm}
    The partial derivatives of $\mathcal{L}(\mathbf{w})$ are given by:
    \begin{equation*}
        \frac{\partial \mathcal{L}(\mathbf{w})}{\partial w_k} =
        \frac{2}{N} 
        \sum_{i=1}^{N} 
        \Big(y^{(i)} - \sum_{j=0}^{n} x_j^{(i)} w_j\Big)\Big(-x_k^{(i)}\Big) \Rightarrow
    \end{equation*}
    \begin{equation}
        \frac{\partial \mathcal{L}(\mathbf{w})}{\partial w_k} =
        \frac{2}{N} 
        \Big\{             
        \sum_{i=1}^{N} \sum_{j=0}^{n} x_k^{(i)} x_j^{(i)} w_j -
        \sum_{i=1}^{N} x_k^{(i)} y^{(i)}  
        \Big\}
        \label{eq:linear_regression_closed_form_solution_derivative_calculation}
    \end{equation}

    \framebreak

    From Eqs.~\ref{eq:linear_regression_closed_form_solution_zero_derivative} and
    \ref{eq:linear_regression_closed_form_solution_derivative_calculation}, 
    we obtain:
    \begin{equation}
        \sum_{i=1}^{N} \sum_{j=0}^{n} x_k^{(i)} x_j^{(i)} w_j -
        \sum_{i=1}^{N} x_k^{(i)} y^{(i)} = 0
        \label{eq:linear_regression_closed_form_solution_coupled_equations_1}
    \end{equation}
    which can be re-arranged as:
    \begin{equation}
        \sum_{j=0}^{n} \Big(\sum_{i=1}^{N} x_k^{(i)} x_j^{(i)} \Big) w_j = 
        \sum_{i=1}^{N} x_k^{(i)} y^{(i)}  
        \label{eq:linear_regression_closed_form_solution_coupled_equations_2}
    \end{equation}

    Eq.~\ref{eq:linear_regression_closed_form_solution_coupled_equations_2} 
    describes a system of $n$+1 linear equations, one for each $k \in [0,n]$,
    coupling $n$+1 real-valued variables $\mathbf{w}=(w_0=b, w_1,...w_{n})^T$.

    \framebreak

    This becomes more apparent, if we rewrite 
    Eq.~\ref{eq:linear_regression_closed_form_solution_coupled_equations_2} as:
    \begin{equation}
        \sum_{j=0}^{n} A_{kj} w_j = c_k
        \label{eq:linear_regression_closed_form_solution_coupled_equations_3}
    \end{equation}\\
    \vspace{-0.1cm}
    where:\\
    \vspace{-0.5cm}
    \begin{equation}
        A_{kj} = \sum_{i=1}^{N}  x_k^{(i)} x_j^{(i)}
        \label{eq:linear_regression_closed_form_solution_coupled_equations_A}
    \end{equation}\\
    \vspace{-0.3cm}
    and:\\
    \vspace{-0.3cm}
    \begin{equation}
        c_{k} = \sum_{i=1}^{N} y^{(i)} x_k^{(i)}
        \label{eq:linear_regression_closed_form_solution_coupled_equations_c}
    \end{equation}

    It is straightforward to solve that coupled system of linear equations.\\
    \vspace{0.3cm}
    There are several linear algebra libraries to allow you to solve that 
    system in a computer (e.g. {\tt GSL} or {\tt NumPy}).\\

    \framebreak

    We are going to reformulate the previous expressions
    (Eqs.~\ref{eq:linear_regression_closed_form_solution_coupled_equations_3},
    \ref{eq:linear_regression_closed_form_solution_coupled_equations_A},
    \ref{eq:linear_regression_closed_form_solution_coupled_equations_c}) 
    in terms of {\bf operations of matrices and vectors}.\\
    \vspace{0.2cm}
    In CS, \index{vectorization}\gls{vectorization}
    (or \index{array programming}\gls{array programming}) 
    refers to programming solutions that 
    {\bf allow operations to an entire set of values at once}.\\

    \begin{blockexample}{}
        \small \tt
        import numpy as np\\
        \vspace{0.4cm}
        A = np.random.rand(10,10)\\
        B = A*A + 5.3*A + np.log10(A) + A*np.tanh(A)\\
    \end{blockexample}  

    Notice the absence of a {\em `for loop'} over the matrix elements.
    Not only this produces more readable code, 
    but it can be {\bf very efficient computationally}.
    \begin{itemize}
        \item Fundamentally, matrix multiplication is {\bf highly parallelizable}. 
        \item \gls{ml} frameworks such as {\tt PyTorch} and {\tt TensorFlow} 
              code can run vectorized code very efficiently on \glspl{gpu}.
    \end{itemize}

    \framebreak

    The training set $\mathbb{D}$ contains $N$ examples 
    $(\mathbf{x}^{(i)},y^{(i)})$, with $i \in [1,N]$,
    where, for the $i^{th}$ example, 
    $\mathbf{x}^{(i)}=(x_0^{(i)}=+1, x_1^{(i)},...x_{n}^{(i)})^T$.\\
    \vspace{0.2cm}
        
    All the inputs $\mathbf{x}^{(i)}$ can be packed into a 2-D array $\vect{X}$,
    where each example is a new row and, therefore, $X_{ij} = x_{j-1}^{(i)}$.
    Similarly, all target values $y^{(i)}$ can be packed into a column vector 
    with a new example in each row:\\
    \vspace{-0.2cm}
    \begin{equation}
        \vect{X} =
        \begin{pmatrix}
            x_0^{(1)} &  x_1^{(1)} & ... & x_n^{(1)} \\
            x_0^{(2)} &  x_1^{(2)} & ... & x_n^{(2)} \\
            ...       &  ...       & ... & ...       \\
            x_0^{(N)} &  x_1^{(N)} & ... & x_n^{(N)} \\
        \end{pmatrix} 
        \;\;\; \textrm{and} \;\;\;
        \vect{y} =
        \begin{pmatrix}
            y^{(1)} \\
            y^{(2)} \\
            ...     \\
            y^{(N)} \\
        \end{pmatrix} 
        \label{eq:linear_regression_closed_form_solution_matrix_inputs}
    \end{equation}

    Correspondingly, our linear model can produce the predictions $\hat{y}^{(i)}$ 
    for all training instances $i \in [1,N]$, with a single matrix operation:\\
    \vspace{-0.3cm}
    \begin{equation}
        \vect{\hat{y}} = \vect{X}{\vect{w}}
        \label{eq:linear_regression_closed_form_solution_matrix_output}
    \end{equation}
    \vspace{-0.2cm}
    where $\mathbf{w}=(w_0=b, w_1,...w_{n})^T$.\\

    \framebreak

    Using the matrices $\vect{X}$ and $\vect{y}$ defined in 
    Eq.~\ref{eq:linear_regression_closed_form_solution_matrix_inputs},
    we can define matrices that contain all the elements 
    $A_{kj}$ (Eq.~\ref{eq:linear_regression_closed_form_solution_coupled_equations_A}) and
    $c_{k}$ (Eq.~\ref{eq:linear_regression_closed_form_solution_coupled_equations_c}).

    \begin{equation}
        \vect{A} =         
        \begin{pmatrix}
            x_0^{(1)} &  x_0^{(2)} & ... & x_0^{(N)} \\
            x_1^{(1)} &  x_1^{(2)} & ... & x_1^{(N)} \\
            ...       &  ...       & ... & ...       \\
            x_n^{(1)} &  x_n^{(2)} & ... & x_n^{(N)} \\
        \end{pmatrix} 
        \begin{pmatrix}
            x_0^{(1)} &  x_1^{(1)} & ... & x_n^{(1)} \\
            x_0^{(2)} &  x_1^{(2)} & ... & x_n^{(2)} \\
            ...       &  ...       & ... & ...       \\
            x_0^{(N)} &  x_1^{(N)} & ... & x_n^{(N)} \\
        \end{pmatrix} =
        \vect{X}^{T} \vect{X}
        \label{eq:linear_regression_closed_form_solution_coupled_equations_A_matrix}
    \end{equation}

    \begin{equation}
        \vect{c} =  
        \begin{pmatrix}
            x_0^{(1)} &  x_0^{(2)} & ... & x_0^{(N)} \\
            x_1^{(1)} &  x_1^{(2)} & ... & x_1^{(N)} \\
            ...       &  ...       & ... & ...       \\
            x_n^{(1)} &  x_n^{(2)} & ... & x_n^{(N)} \\
        \end{pmatrix} 
        \begin{pmatrix}
            y^{(1)} \\
            y^{(2)} \\
            ...      \\
            y^{(N)} \\
        \end{pmatrix} = 
        \vect{X}^{T} \vect{y}
        \label{eq:linear_regression_closed_form_solution_coupled_equations_c_matrix}
    \end{equation}

    \framebreak

    Using Eqs.~\ref{eq:linear_regression_closed_form_solution_coupled_equations_A_matrix} and
    \ref{eq:linear_regression_closed_form_solution_coupled_equations_c_matrix},
    Eq.~\ref{eq:linear_regression_closed_form_solution_coupled_equations_3}
    can be written in a matrix form as:
    \begin{equation}
        \vect{A}\vect{w} = \vect{c} \Rightarrow
        \label{eq:linear_regression_closed_form_solution_coupled_equations_matrix_1}
    \end{equation}        
    \begin{equation}
        \vect{X}^{T} \vect{X} \vect{w} = \vect{X}^{T} \vect{y}
        \label{eq:linear_regression_closed_form_solution_coupled_equations_matrix_2}
    \end{equation}        

    The above matrix equation can be easily solved for $\vect{w}$,
    if $\vect{X}^{T} \vect{X}$ can be inverted:\\
    \begin{equation}
        \vect{w} = (\vect{X}^T \vect{X})^{-1} \vect{X}^T \vect{y}
        \label{eq:linear_regression_closed_form_solution_matrix}
    \end{equation}        

\end{frame}

